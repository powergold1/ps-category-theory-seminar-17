\def\pathToRoot{../../}\documentclass[a4paper,oneside]{scrbook}
\usepackage[dvipsnames]{xcolor}
\input{\pathToRoot headers/sharedHeader}

\usepackage{bm}
\usepackage{upgreek}
\usepackage{titlesec}

\geometry{margin=3cm}

\titleformat{\chapter}[display]
{\normalfont\LARGE\bfseries}{\chaptertitlename~\thechapter}{4pt}{\LARGE}
\titlespacing*{\chapter}{0pt}{2pt}{10pt}



% Environments theorem, lemma, definition and corollary.
% They all use the counter from theorem, except remark which has no counter
\newtheorem{theorem}{Theorem}[chapter]
\newtheorem{lemma}[theorem]{Lemma}
\newtheorem{corollary}[theorem]{Corollary}
\newtheorem{fact}[theorem]{Fact}
\newtheorem{example}[theorem]{Example}
\theoremstyle{definition}
\newtheorem{definition}[theorem]{Definition}
\newtheorem*{remark}{Remark}

\usepackage{charter}
\fontfamily{bch}\selectfont



%%% Local Variables:
%%% mode: latex
%%% TeX-master: "../papers/talk-7/paper-7"
%%% End:


\title{General limits}
\author{Leonhard Staut}

\begin{document}

%% TODO: consistent notation, DI D(I) D_I, also for maps D_m Dm

\maketitle
\section*{Introduction}
We have seen three special structures in categories,
the product, the pullback, and the equalizer.
As an example, the binary product is constructed in the following way.
We start with two objects. Then an object $U$ with maps to these objects is the product,
if it exists an satisfies the universal property for products, that is, all other such objects $V$
factor through $U$ uniquely.\\[1em]
\begin{minipage}{.35\linewidth}
  \begin{tikzcd}
    \bullet &&\bullet
  \end{tikzcd}
\end{minipage}%
\begin{minipage}{.35\linewidth}
  \begin{tikzcd}
    & U \arrow{dl} \arrow{dr} &\\
    \bullet &&\bullet
  \end{tikzcd}
\end{minipage}%
\begin{minipage}{.35\linewidth}
  \begin{tikzcd}
    & V \arrow[bend right=40]{ddl} \arrow[bend left=40]{ddr} \arrow[dashed]{d} &\\
    & U \arrow{dl} \arrow{dr} &\\
    \bullet &&\bullet
  \end{tikzcd}
\end{minipage}
\\[1em]
The pullback and the equalizer are constructed similarly.
They all have an input diagram, and we call an object $U$ with maps into this diagram
product/pullback/equalizer if it satisfies the universal property.
\\[1em]
\begin{minipage}{.35\linewidth}
  \begin{tikzcd}
    & V \arrow[bend right=40]{ddl} \arrow[bend left=40]{ddr} \arrow[dashed]{d} &\\
    & U \arrow{dl} \arrow{dr} &\\
    \bullet &&\bullet
  \end{tikzcd}
\end{minipage}%
\begin{minipage}{.35\linewidth}
  \begin{tikzcd}
    V
    \arrow[bend left=40]{drr}
    \arrow[bend right=40]{ddr}
    \arrow[dashed,swap]{dr}
    && \\
    &U
    \arrow{r}
    \arrow{d}
    &\bullet \arrow{d} \\
    & \bullet \arrow{r} & \bullet
  \end{tikzcd}
\end{minipage}%
\begin{minipage}{.3\linewidth}
  \begin{tikzcd}
    V \arrow[dashed]{dd} \arrow{dr}&& \\
    & \bullet \arrow[shift left]{r} \arrow[shift right]{r} & \bullet \\\
    U \arrow{ur} &&
  \end{tikzcd}
\end{minipage}\\[2em]
%% TODO: possibly give further applications of limits too
%% TODO: maybe some more intuition what limits are about

\section{Definition}
Products, pullbacks, and equalizers are instances of a more general concept called
limit which expresses the common pattern between the examples above.
The goal is to make this abstract common pattern more formal
and establish relations between concrete instances of it.
We start with the definition of a limit.\\

\begin{definition}
  A \textbf{limit} is a universal cone over a diagram.
\end{definition}

In the following section we explain this definition. We need to clarify three things
in particular.
First we need to define the concept of a diagram more formally.
Then we define what a cone over a diagram is. Lastly we state what it means for a cone
to be universal.
We start with diagrams. They are the initial entities of which we take the limit.
In the examples above the diagrams
are the various collections of objects (and arrows) that we start with
and over which the object $U$ has maps.

\begin{definition}
  Let $\bm{I}, \mathscr{C}$ be categories.
  A \textbf{diagram} is a functor $D \from \bm{I} \to \mathscr{C}$.
  The category $\bm{I}$ is called the \textbf{shape} of the diagram.
\end{definition}
\begin{example}
  \label{diagramexample}
  This is an example of a diagram we use to build a pullback. \\[1cm]
  \begin{minipage}{0.3\linewidth}
    \hspace{2cm}
    \begin{tikzcd}
      & \bullet \arrow{d} \\
      \bullet\arrow{r} & \bullet
    \end{tikzcd}\\
  \end{minipage}%
  \begin{minipage}{0.2\linewidth}
    $\qquad\quad\overset{D}{\Longrightarrow}$
  \end{minipage}%
  \begin{minipage}{0.1\linewidth}
    \begin{tikzpicture}
      \draw(1,0) circle (1.5cm);
      \begin{tikzcd}
        & \bullet \arrow{d} \\
        \bullet\arrow{r} & \bullet
      \end{tikzcd}
    \end{tikzpicture}
  \end{minipage}\\[2em]
  \begin{minipage}{0.3\linewidth}
    \hspace{2cm}
    $\qquad \bm I$
  \end{minipage}%
  \begin{minipage}{0.2\linewidth}
    \
  \end{minipage}%
  \begin{minipage}{0.3\linewidth}
    \hspace{2cm}
    $\qquad \mathscr C$
  \end{minipage}\\[1em]
  We did not give names to the objects as we are not concerned with them.
\end{example}

The functor $D$ yields
a collection of objects and morphisms in the category $\mathscr C$
that looks just like $\bm I$.
We also say $D$ \textbf{identifies} diagrams of shape $\bm I$.
The shape category has no further meaning. Its only purpose
is to define how the diagram is supposed to look like.
The reason we use an intermediate
category $\bm I$ to define the shape is that
there might be many parts in $\mathscr C$ which have the shape $\bm I$,
especially when $\bm I$ is so small.
We therefore get all different diagrams of that shape by considering
all different functors $D$.
Diagrams do not necessarily have to be small. Their shape can even be infinite.
However most interesting diagrams tend to be very small.\\
We are typically interested in diagrams of a specific category,
$\mathscr C$ in the example. Therefore we usually leave out $\bm I$ and $D$,
and we just draw the identified part of $\mathscr C$ instead.\\
The next part on the way to the limit definition is the concept of a cone which
is what we address now.
\begin{definition}
  Let $D \from \bm{I} \to \mathscr{C}$ be a diagram.
  A \textbf{cone} $U$ over $D$ is an object $U \in \mathscr{C}$
  with maps $U \overset{f_I}{\rightarrow} D_I, \forall\ I \in \bm{I}$, such that
  $\forall\ I \overset{m}{\to} J \in \bm{I}$, the following triangle commutes:
  \[
    \begin{tikzcd}
      & U \arrow[dl, swap, "f_I"] \arrow{dr}{f_J}&\\
      D_I \arrow[rr, swap, "D_m"] & & D_J\\
    \end{tikzcd}
  \]
  That is $f_J = D(m) \circ f_I$.\\
  For a cone we write $(U \overset{f_I}{\to} D(I))_{I\in \bm{I}}$.
  We call $U$ the \textbf{vertex}, and $f_I$ the \textbf{projection maps} of the cone.
\end{definition}
\begin{remark}
  When it is clear from the context that $U$ is the vertex of a cone, then we use just
  the vertex $U$
  to refer to the whole cone.\\
  Note that the underlying diagram determines how the cone looks like.
\end{remark}
%% TODO: Notation DI, D(I), D_I ??
\begin{example}
  A cone over the diagram of pullbacks
  (\hyperref[diagramexample]{Example \ref*{diagramexample}}) looks like this.
  \[
    \begin{tikzcd}
      U \arrow[bend right=30]{ddr}{f_I} \arrow[bend left=30]{drr}{f_k} \arrow[bend left=20, swap]{ddrr}{f_J}&   &\\
      & & D_K \arrow{d}{D_n} \\
      & D_I \arrow{r}{D_m} & D_J
    \end{tikzcd}
  \]
  The commuting condition says that $D_m \circ f_I = f_J$ and $D_n \circ f_K = f_J$.
\end{example}

In other words a cone is an object that covers the diagram with projection maps,
such that all triangles commute.\\
The last step we need for the
general limit definition is
to define what a universal cone is.

\begin{definition}
  A cone $(U \overset{f_I}{\to} D(I))_{I\in \bm{I}}$ is \textbf{universal}, if
  for all other cones $(V \overset{g_I}{\to} D(I))_{I\in \bm{I}}$ there is a unique
  map $h \from V \to U$, such that for all $I \in \bm{I}$
  the following triangles commute.
  \[
    \begin{tikzcd}
      V \arrow[dr, swap, "g_I"] \arrow{r}{h} & U \arrow{d}{f_I} \\
      & D_I
    \end{tikzcd}
  \]
  That is $g_i = f_i \circ h$.
\end{definition}

Intuitively, a cone is universal if it
is universally the ``closest'' to the diagram.

\begin{example}
  This is an example of a universal cone over the
  diagram of pullbacks
  (\hyperref[diagramexample]{Example \ref*{diagramexample}}).
    \[
    \begin{tikzcd}
      {\color{red}V}
      \arrow[bend right=40,color=red]{dddrr}{g_I}
      \arrow[bend left=40, color=red]{ddrrr}{g_K}
      \arrow[bend left=30, color=red, swap]{dddrrr}{g_J}
      \arrow[dashed,swap]{dr}{h}
      & & & \\
      & {\color{blue}U}
      \arrow[bend right=30,color=blue]{ddr}{f_I}
      \arrow[bend left=30, color=blue]{drr}{f_K}
      \arrow[bend left=20, color=blue, swap]{ddrr}{f_J}&   &\\
      & & & D_K \arrow{d}{D_n} \\
      & & D_I \arrow{r}{D_m} & D_J
    \end{tikzcd}
  \]
  The universal cone {\color{blue}U}
  is drawn in blue and the other cone {\color{red}V} is drawn in red.
  Both cones obviously have to satisfy the commuting condition for cones.
  Additionally the commuting condition for universal cones states that
  $g_I = f_I \circ h$, $g_J = f_J \circ h$, and $g_K = f_K \circ h$.
  Therefore this example commutes in its entirety.\\
  Note that there could be several other cones like $V$.
\end{example}

A limit is therefore characterized by the kind of diagram
over which it is a universal cone, meaning we get different instances of limits
by choosing diagrams of different shapes.

\begin{example}
  Consider a diagram of the following shape
  \begin{tikzcd}
    \bullet &\bullet
  \end{tikzcd},
  which is just two objects without non-trivial arrows.
  A limit of this diagram looks like this.
  \[
    \begin{tikzcd}
      & {\color{red} V}
      \arrow[bend right=40,color=red, swap]{ddl}{g_I}
      \arrow[bend left=40,color=red]{ddr}{g_J}
      \arrow[dashed]{d}{h} &\\
      & {\color{blue} U}
      \arrow[color=blue, swap]{dl}{f_I}
      \arrow[color=blue]{dr}{f_J} &\\
      I && J
    \end{tikzcd}
  \]
  This is the ordinary binary product.
  The commuting condition for limits states that
  $f_I \circ h = g_I$, and $f_J \circ h = g_J$, which
  is exactly the commuting condition for products.
  We have therefore verified that the product is
  indeed an instance of the general limit definition.
\end{example}

\begin{example}
  \label{terminallimit}
  If we take the empty category as the shape for a diagram, then a cone
  over that diagram is just a single object $U$ that has no other required arrows.
  The universal property for limits then states that every other cone, i.e. any object $V$,
  has a unique map $V \to U$. A limit of the empty diagram is therefore just a terminal object.
\end{example}

\begin{example}
  Consider the following diagram.
  \begin{tikzcd}
    I \arrow{r}{m} & J & K \arrow[swap]{l}{n}
  \end{tikzcd}
  A cone over this diagram looks like this:
  \[
    \begin{tikzcd}
      U
      \arrow[]{r}{\ \ \ f_K}
      \arrow[swap]{rd}{\ f_J}
      \arrow[swap]{d}{f_I}
      &K \arrow{d}{n} \\
      I \arrow[swap]{r}{m} & J
    \end{tikzcd}
  \]
  The commuting condition for cones states however that
  $f_J = m\circ f_I$, and $f_J = n\circ f_K$. Since $f_J$ is uniquely determined
  by composition of other arrows, we do not need to draw it and leave it implicit here
  and for all other cones over this diagram.
  With this in mind a limit of this diagram looks like this:
  \[
    \begin{tikzcd}
      {\color{red}V}
      \arrow[bend left=50,color=red]{drr}{g_K}
      \arrow[bend right=50,color=red,swap]{ddr}{g_I}
      \arrow[dashed,swap]{dr}{h}
      && \\
      &{\color{blue}U}
      \arrow[color=blue]{r}{f_K}
      \arrow[color=blue, swap]{d}{f_I}
      &K \arrow{d}{n} \\
      & I \arrow[swap]{r}{m} & J
    \end{tikzcd}
  \]
  The commuting condition for $U$ being a limit is exactly the commuting condition for
  $U$ being a pullback.
  Thus pullbacks are a valid instance of the general limit definition.
  It follows similarly that equalizers are limit instances over diagrams of the shape
  \begin{tikzcd}
    I \arrow[r,shift left, "m"] \arrow[r,shift right,swap,"n"] & J
  \end{tikzcd}. For equalizers we use the same approach we used above to ignore the morphism
  $f_J$ to make the limit look like the conventional equalizer.
\end{example}

\section{Building new limits}

\begin{definition}
We say a category \textbf{has limits} over diagrams of shape $\bm I$, if for all
diagrams of shape $\bm I$, there exists a universal cone over it.
\end{definition}
A question that often arises is if a given category $\mathscr C$ has certain limits,
e.g. products or a terminal object. Now that we have a abstract definition of limits
we can approach this problem in a more systematic way.
This leads to two observations:\\
\hyperref[equiinstance]{(\ref*{equiinstance})} A category $\mathscr C$ has finite products and equalizers iff
it has pullbacks and a terminal
object.\\
\hyperref[arbitrarylimits]{(\ref*{arbitrarylimits})}
If a category $\mathscr C$ has products and equalizers, then it has all limits, that is,
limits of diagrams of arbitrary shape can be constructed using these two instances.

\subsection{Products}
So far we have only been concerned with the product of two objects, i.e. the binary product.
We now study the concept of products more generally.
\begin{lemma}
  \label{emptyproduct}
  The empty product, i.e. the product of zero objects, is a terminal objects
\end{lemma}
\begin{proof}
  The empty product is a limit over the empty diagram just as the terminal object
  as discussed in
  \hyperref[terminallimit]{Example \ref*{terminallimit}}
\end{proof}

Using discrete categories with more than two objects
we get products of arbitrary size.
\begin{example}
  This is an example of a product of $k$ objects. We use the symbol $\prod$ for
  arbitrary products.
  \[
    \begin{tikzcd}
      && {\color{red} V}
      \arrow[bend right=40,color=red]{ddll}
      \arrow[bend right=40,color=red]{ddl}
      \arrow[bend left=40,color=red]{ddrr}
      \arrow[bend left=40,color=red]{ddr}
      \arrow[dashed]{d} &&\\
      && {\color{blue} {\displaystyle \prod_{i \in \{1,\ldots,k\} } D_i}}
      \arrow[color=blue]{dl}
      \arrow[color=blue]{dll}
      \arrow[color=blue]{dr}
      \arrow[color=blue]{drr} &&\\
      D_1 & D_2 & \ldots & D_{k-1} & D_k
    \end{tikzcd}
  \]
\end{example}

\begin{lemma}
  \label{finiteproducts}
  Let $\mathscr C$ be a category with a terminal object $1$ and binary products.
  Then we can construct all finite products.
\end{lemma}
\begin{proof}
  \begin{itemize}
  \item The empty product is just the terminal object by
    \hyperref[emptyproduct]{Example \ref*{emptyproduct}}
  \item The unary product of any object $A$ is constructed by the binary product
    $A\times 1$ %% TODO: explain why this works...
  \item The binary product exists by assumption
  \item We want to build some finite product over objects $D_1, \dots D_k$.
    Assume by induction that the following two products exist.
    \[ P_1 \coloneqq \displaystyle \prod_{i \in \{1, \dots, m \} } D_i \text{ , and } P_2 \coloneqq \displaystyle \prod_{i \in \{m+1, \dots, k\} } D_i \]
    with projection maps $(\pi_i)_{i \in \{1, \dots, m \} }$ for $P_1$, and $(\tau_i)_{i \in \{m+1, \dots, k\} }$ for $P_2$.
    Then there exists a binary product of $P_1 \times P_2$ with projection maps $f_{P_1}, f_{P_2}$.
    $P_1 \times P_2$ has maps $\pi_i \circ f_{P_1}$ and $\tau_i \circ f_{P_2}$.
    $P_1 \times P_2$ is therefore a cone.
    Since $P_1$ and $P_2$ are products, any other cone $V$ over them has unique maps $h_{P_1}$ and $h_{P_2}$.
    The unique morphism $V \rightarrow P_1 \times P_2$ is given by $\langle h_{P_1}, h_{P_2} \rangle$.
    $P_1 \times P_2$ is therefore universal.
  \end{itemize}
  Intuitively the construction works by stacking binary products on top of each other.
  Since we assume $\mathscr C$ to have all products, the product of two products exists. This way we can create
  arbitrarily large finite products.
\end{proof}

\subsection{Equivalence of instances}
\label{equiinstance}

In this section we establish that the instances of limits we have seen are related
in a way that allows us to show how to construct these instances using others.
This implies that if we know that a category has some of the right instances
we can construct other limits for free using only category theoretical arguments.

\begin{theorem}
  \label{pullbackconstruction}
  Let $\mathscr C$ be a category with finite products and equalizers.
  Then we can construct pullbacks.
\end{theorem}
\begin{proof}
We start with the diagram for pullbacks.
\[
  \begin{tikzcd}
    & K \arrow{d}{n} \\
    I \arrow[swap]{r}{m} & J
  \end{tikzcd}
\]
We need to construct a limit of this diagram.
First we take the binary product of $I$ and $K$ which exists by assumption.
\[
  \begin{tikzcd}
    I \times K \arrow{r}{p_K} \arrow{d}{p_I} & K \arrow{d}{n} \\
    I \arrow[swap]{r}{m} & J
  \end{tikzcd}
\]
This looks already similar to a pullback.
However, we need to ensure the commuting condition.
For this purpose we use an equalizer which also exists by assumption.
\[
  \begin{tikzcd}
    E \arrow{r}{e} &
    I \times K
    \arrow[r,shift right,swap, "m \circ p_I"] \arrow[r,shift left, "n \circ p_K"]
    & J \\
  \end{tikzcd}
\]
We now unfold this drawing.
\[
  \begin{tikzcd}
    E \arrow{dr}{e}& &\\
    & I \times K \arrow[swap]{d}{p_I} \arrow{r}{p_K}& K \arrow{d}{n}\\
    & I \arrow[swap]{r}{m} & J
  \end{tikzcd}
  \qquad \bm{\rightsquigarrow} \qquad 
  \begin{tikzcd}
    E \arrow[swap]{d}{p_I \circ e} \arrow{r}{p_K \circ e} & K \arrow{d}{n} \\
    I \arrow[swap]{r}{m} & J
  \end{tikzcd}
\]
$E$ satisfies the commuting and universality conditions of limits
since it is already an equalizer.
%% TODO: more detail...
Therefore it is a pullback.
\end{proof}

\begin{fact}
  \label{pullbackproduct}
  Let $\mathscr C$ be a category with pullbacks and a terminal object $1$.
  Then we can construct finite products.
\end{fact}
\begin{proof}
Finite products can be constructed
with the terminal object and binary products by
\hyperref[finiteproducts]{Lemma \ref*{finiteproducts}}.
Therefore we only show how to construct binary products.
Consider following the diagram:
\[
  \begin{tikzcd}
    & B \arrow{d} \\
    A \arrow{r} & 1
  \end{tikzcd}
\]
The pullback $U$ exists by assumption.
\[
  \begin{tikzcd}
    U \arrow{r}{p_B} \arrow{d}{p_A} & B \arrow{d}{f} \\
    A \arrow{r}{g} & 1
  \end{tikzcd}
\]
The arrows $f$ and $g$ are unique and therefore monic.
Hence $U$ satisfies the universality condition and is indeed the product.
%% TODO: possibly show this for one calculation
\end{proof}

Now that we know that a category with pullbacks and a terminal object
also has finite products, we can proceed to construct equalizers as well
as their construction build on the existence of products in this category.

\begin{theorem}
  Let $\mathscr C$ be a category with pullbacks and a terminal object $1$.
  Then we can construct equalizers.
\end{theorem}
\begin{proof}
  We start with the initial diagram for equalizers.
  \[
    \begin{tikzcd}
      A \arrow[shift left]{r}{f} \arrow[shift right, swap]{r}{g} & B
    \end{tikzcd}
  \]
  We transform this into the following diagram.
  %%TODO: explain why they're equivalent
  \[
    \begin{tikzcd}
      & B \arrow{d}{\langle 1_B, 1_B \rangle} \\
      A \arrow{r}{\langle f,g \rangle} & B \times B
    \end{tikzcd}
  \]
  The product $B \times B$ exists by \hyperref[pullbackproduct]{Theorem \ref*{pullbackproduct}}.
  We then have the pullback $E$, such that the following diagram commutes.
  \[
    \begin{tikzcd}
      E \arrow{r}{h} \arrow{d}{e} & B \arrow{d}{\langle 1_B, 1_B \rangle} \\
      A \arrow{r}{\langle f,g \rangle} & B \times B
    \end{tikzcd}
  \]
  Since $E$ is a pullback the square above commutes and
  we get $f \circ e = 1_B \circ h = h = 1_B \circ h = g \circ e$, and $E$ is therefore
  indeed an equalizer of $f$ and $g$.
\end{proof}

%%TODO: fix this section
\subsection{Construction of arbitrary limits}
\label{arbitrarylimits}
We now study how arbitrary limits can be constructed using only a
few specific instances.
The approach is similar to the construction of pullbacks in
\hyperref[pullbackconstruction]{Fact \ref*{pullbackconstruction}}.

\begin{fact}
  Let $\mathscr C$ be a category that has all products and equalizers.
  Let $D : \bm I \to \mathscr C$ be any diagram.
  Then we can construct a limit of $D$.
\end{fact}
\begin{proof}
We use the following construction:
\[
  \begin{tikzcd}
    E \arrow{r}{e} &
    \displaystyle \prod_{I \in \bm{I}} D(I)
    \arrow[r,shift left, "s"] \arrow[r,shift right,swap,"t"] &
    \displaystyle \prod_{K \overset{n}{\to} J \in \bm{I}} D(J)
  \end{tikzcd}
\]
with the maps $s, t$ that have components $s_n, t_n$ for all $K \overset{n}{\to} J \in \bm{I}$:
\begin{align*}
  s_n &= Dn \circ p_K\ \hat{=} \prod_{I \in \bm{I}} D(I)
        \overset{p_K}{\rightarrow} D_K \overset{Dn}{\rightarrow} D_J \\
  t_n &= p_J\ \hat{=} \prod_{I \in \bm{I}} D(I)
        \overset{p_J}{\rightarrow} D_J
\end{align*}
All products exist by assumption.
To clarify, the product over all maps $K \overset{n}{\to} J \in \bm{I}$
has components $p_n = J$ for each map $n$.\\
We take $E$ as the equalizer of $s$ and $t$ which also exists by assumption.
We therefore have $s \circ e = t \circ e$.
The maps $s$ and $t$ have components. Therefore the equality has to hold for all
components and we get
\begin{align*}
  s \circ e &= t \circ e\\
  s_n \circ e &= t_n \circ e, \quad \forall\ K \overset{n}{\rightarrow} J \in \bm I \\
  Dn \circ p_K \circ e &= p_J \circ e
\end{align*}
The goal is to construct a limit.
We do this by constructing a cone and showing that it is universal.
For the construction of a cone
we need a vertex and projection maps into every object of the diagram,
such that the commuting condition of cones is satisfied.
We take $E$ as the vertex and
$\uprho_I \coloneqq p_I \circ e, \forall\ I \in \bm I$
as the projection maps of the cone.
Clearly, we have all necessary projection maps and we only need to verify
the commuting condition of cones.
We have all maps $K \overset{n}{\rightarrow} J \in \bm I$
\begin{align*}
  Dn \circ p_K \circ e &= p_J \circ e \\
  Dn \circ \uprho_K &= \uprho_J
\end{align*}
which corresponds to the commuting diagram
\[
  \begin{tikzcd}
    E \arrow{r}{\uprho_K} \arrow{dr}[swap]{\uprho_J} & D_K \arrow{d}{Dn}\\
    & D_J
  \end{tikzcd}
\]
This is exactly the commuting condition we need for a cone.\\
Since any other cone over $D$
is also a fork of $s$ and $t$
it follows that the equalizer $E$ is also a universal cone
and therefore a limit of $D$.
\end{proof}
\begin{remark}
  We can even construct limits of infinite diagrams
  because we assumed that $\mathscr C$ has all products.
  If $\mathscr C$ only has finite products, then we can
  use this construction for
  limits of finite diagrams.\\
  Using \hyperref[equiinstance]{Section \ref*{equiinstance}}
  we can alternatively require pullbacks and a terminal object instead of finite products.
\end{remark}

%%TODO: fix this section
\section{Colimits}
To define colimits we dualize the concept of limits.
\begin{definition}
  Let $D : \bm I \to \mathscr C$ be a diagram.
  A \textbf{cocone} $U$ over $D$ is an object $U \in \mathscr{C}$
  with maps $D_I \overset{ f_I}{\rightarrow} U,\ \forall\ I \in \bm{I}$, such that
  $\forall\ I \overset{m}{\to} J \in \bm{I}$, the following triangle commutes:
  \[
    \begin{tikzcd}
      D_I \arrow[rr, "D(m)"]\arrow[dr, swap, "f_I"] & & D_J \arrow{dl}{f_J}\\
      & U  &
    \end{tikzcd}
  \]
  That is $f_I = f_J \circ D(m)$.
\end{definition}
A cone has maps from the vertex into the diagram,
while a cocone has maps from the diagram into the vertex.
\begin{definition}
  A cocone $(D(I)\overset{f_I}{\to} U)_{I\in \bm{I}}$ is \textbf{universal}, if
  for all other cones $(D(I) \overset{g_I}{\to} V)_{I\in \bm{I}}$ there is a unique
  map $h \from U \to V$, such that for all $I \in \bm{I}$
  the following triangles commute.
  \[
    \begin{tikzcd}
      D_I \arrow[dr, swap, "g_I"] \arrow{r}{f_I} & U \arrow{d}{h} \\
      & V
    \end{tikzcd}
  \]
  That is $g_i = f_i \circ h$.
\end{definition}

\begin{definition}
  A \textbf{colimit} is a universal co-cone over a diagram.
\end{definition}
%% TODO: complete definition

\[
    \begin{tikzcd}
      & D_K \arrow{d}{D_n}
      \arrow[bend left=30,color=blue]{ddr}{f_K}
      \arrow[bend left=50,color=red]{dddrr}{g_K} & & \\
      D_I \arrow{r}{D_m}
      \arrow[bend right=30,color=blue]{drr}{f_I}
      \arrow[bend right=40,color=red]{ddrrr}{g_I} &
      D_J \arrow[bend left=20,swap, color=blue]{dr}{f_J}
      \arrow[bend left=40,color=red]{ddrr}{g_J} & & \\
      & & {\color{blue}U} \arrow[dashed]{dr}{h}& \\
      & & & {\color{red}V}
    \end{tikzcd}
\]
%% TODO: example drawing

\end{document}

%%% Local Variables:
%%% mode: latex
%%% TeX-master: t
%%% End:
