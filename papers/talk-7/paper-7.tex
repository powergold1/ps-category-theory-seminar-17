\def\pathToRoot{../../}\documentclass[a4paper,oneside]{scrbook}
\usepackage[dvipsnames]{xcolor}
\input{\pathToRoot headers/sharedHeader}

\usepackage{bm}
\usepackage{upgreek}
\usepackage{titlesec}

\geometry{margin=3cm}

\titleformat{\chapter}[display]
{\normalfont\LARGE\bfseries}{\chaptertitlename~\thechapter}{4pt}{\LARGE}
\titlespacing*{\chapter}{0pt}{2pt}{10pt}



% Environments theorem, lemma, definition and corollary.
% They all use the counter from theorem, except remark which has no counter
\newtheorem{theorem}{Theorem}[chapter]
\newtheorem{lemma}[theorem]{Lemma}
\newtheorem{corollary}[theorem]{Corollary}
\newtheorem{fact}[theorem]{Fact}
\newtheorem{example}[theorem]{Example}
\theoremstyle{definition}
\newtheorem{definition}[theorem]{Definition}
\newtheorem*{remark}{Remark}

\usepackage{charter}
\fontfamily{bch}\selectfont



%%% Local Variables:
%%% mode: latex
%%% TeX-master: "../papers/talk-7/paper-7"
%%% End:


\title{General limits}
\author{Leonhard Staut}

\begin{document}

\maketitle
\chapter{Introduction}
We have seen examples of limits.\\
Insert examples here.\\
Now we generalize them...
\chapter{Definition}

In category theory we take limits over diagrams.
We begin by formally defining what a diagram is.

\begin{definition}
  Let $\bm{I}, \mathscr{C}$ be categories.
  A \textbf{diagram} is a functor $D \from \bm{I} \to \mathscr{C}$.
  The category $\bm{I}$ is called the \textbf{shape} of the diagram.
\end{definition}
\begin{example}
  \label{diagramexample}
  This is an example of a diagram of shape $\bm I$. \\[1cm]
  \begin{minipage}{0.3\linewidth}
    \hspace{2cm}
    \begin{tikzcd}
      & \bullet \arrow{d} \\
      \bullet\arrow{r} & \bullet
    \end{tikzcd}\\
  \end{minipage}%
  \begin{minipage}{0.2\linewidth}
    $\qquad\quad\overset{D}{\Longrightarrow}$
  \end{minipage}%
  \begin{minipage}{0.1\linewidth}
    \begin{tikzpicture}
      \draw(1,0) circle (1.5cm);
      \begin{tikzcd}
        & \bullet \arrow{d} \\
        \bullet\arrow{r} & \bullet
      \end{tikzcd}
    \end{tikzpicture}
  \end{minipage}\\[2em]
  \begin{minipage}{0.3\linewidth}
    \hspace{2cm}
    $\qquad \bm I$
  \end{minipage}%
  \begin{minipage}{0.2\linewidth}
    \
  \end{minipage}%
  \begin{minipage}{0.3\linewidth}
    \hspace{2cm}
    $\qquad \mathscr C$
  \end{minipage}\\[1em]
\end{example}

The functor $D$ yields
a part of the category $\mathscr C$ that essentially looks like $\bm I$.
We also say $D$ \textbf{identifies} diagrams of shape $\bm I$,
which is why we use $\bm I$ to denote the shape.
Especially with diagrams of such a small shape
there might be many parts in $\mathscr C$ that have the shape of $\bm I$.
We therefore get all different diagrams of that shape by considering
all different functors $D$.
Diagrams do not need to be small though. Their shape can also be infinite.\\
We are typically interested in diagrams of a specific category,
$\mathscr C$ in the example. Therefore we usually leave out $\bm I$ and $D$,
and we just draw the identified part of $\mathscr C$ instead.

\begin{definition}
  Let $D \from \bm{I} \to \mathscr{C}$ be a diagram.
  A \textbf{cone} $U$ over $D$ is an object $U \in \mathscr{C}$
  with maps $f_I, \forall\ I \in \bm{I}$, such that
  $\forall\ I \overset{m}{\to} J \in \bm{I}$, the following triangle commutes:
  \[
    \begin{tikzcd}
      & U \arrow[dl, swap, "f_I"] \arrow{dr}{f_J}&\\
      D_I \arrow[rr, swap, "D(m)"] & & D_J\\
    \end{tikzcd}
  \]
  That is $f_J = D(m) \circ f_I$.\\
  For a cone we write $(U \overset{f_I}{\to} D(I))_{I\in \bm{I}}$.
  We call $U$ the \textbf{vertex}, and $f_I$ the \textbf{projection maps} of the cone.
\end{definition}
\begin{remark}
  When $U$ is the vertex of a cone, then we sometimes use just $U$ to refer to the whole cone
  for better readability.
\end{remark}
%% TODO: Notation DI, D(I), D_I ??
\begin{example}
  A cone over the diagram of
  \hyperref[diagramexample]{Example \ref*{diagramexample}} looks like this.
  \[
    \begin{tikzcd}
      U \arrow[bend right=30]{ddr}{f_I} \arrow[bend left=30]{drr}{f_k} \arrow[bend left=20, swap]{ddrr}{f_J}&   &\\
      & & D_K \arrow{d}{D_n} \\
      & D_I \arrow{r}{D_m} & D_J
    \end{tikzcd}
  \]
  The commuting condition says that $D_m \circ f_I = f_J$ and $D_n \circ f_K = f_J$.
\end{example}

%% TODO: write a sentence like "In other words, ..." and give example. example might be tricky to draw

Now we define what it means for a cone to be universal. Intuitively,
a cone is universal if it satisfies a property that makes it universal among cones,
specifically, that all other cones factor through it uniquely.

\begin{definition}
  A cone $(U \overset{f_I}{\to} D(I))_{I\in \bm{I}}$ is universal, if
  for all other cones $(V \overset{g_I}{\to} D(I))_{I\in \bm{I}}$ there is a unique
  map $h \from V \to U$, such that for all $I \in \bm{I}$
  the following triangles commute.
  \[
    \begin{tikzcd}
      V \arrow[dr, swap, "g_I"] \arrow{r}{h} & U \arrow{d}{f_I} \\
      & D_I
    \end{tikzcd}
  \]
  That is $g_i = f_i \circ h$.
\end{definition}

In other words, any cone over $D$ has maps into $D$, and a cone $U$ is universal if
for any other cone $V$, you can first go to $U$, that is the unique morphism.
From $U$ you can go to $D_I$, and the commuting condition says that this is equal
to the morphism that immediately takes you from $V$ to $D_I$.

\begin{example}
    \[
    \begin{tikzcd}
      {\color{red}V}
      \arrow[bend right=40,color=red]{dddrr}{g_I}
      \arrow[bend left=40, color=red]{ddrrr}{g_k}
      \arrow[bend left=30, color=red, swap]{dddrrr}{g_J}
      \arrow[dashed,swap]{dr}{h}
      & & & \\
      & {\color{blue}U}
      \arrow[bend right=30,color=blue]{ddr}{f_I}
      \arrow[bend left=30, color=blue]{drr}{f_k}
      \arrow[bend left=20, color=blue, swap]{ddrr}{f_J}&   &\\
      & & & D_K \arrow{d}{D_n} \\
      & & D_I \arrow{r}{D_m} & D_J
    \end{tikzcd}
  \]
  The universal cone {\color{blue}U}
  is drawn in blue and the other cone {\color{red}V} is drawn in red.
  Both cones obviously have to satisfy the commuting condition for cones.
  Additionally the commuting condition for universal cones states that
  $g_I = f_I \circ h$, $g_J = f_J \circ h$, and $g_K = f_k \circ h$.
\end{example}

\begin{definition}
  A limit is a universal cone over a diagram.
\end{definition}

A limit is therefore characterized by the kind of diagram
over which it is a universal cone, meaning we get different instances of limits
by choosing diagrams of different shapes.

%% TODO: terminal object, pullback, equalizer, product.

\chapter{Building new limits}
It is possible to build new limits...
%% TODO: example
%% TODO: general limits
%% TODO: possibly other construction mentioned in awodey.

\chapter{Colimits}
Colimits are the dual of limits.
\begin{definition}
  A \textbf{colimit} is a universal co-cone over a diagram.
\end{definition}
%% TODO: complete definition

\[
    \begin{tikzcd}
      & D_K \arrow{d}{D_n}
      \arrow[bend left=30,color=blue]{ddr}{f_K}
      \arrow[bend left=50,color=red]{dddrr}{g_K} & & \\
      D_I \arrow{r}{D_m}
      \arrow[bend right=30,color=blue]{drr}{f_I}
      \arrow[bend right=40,color=red]{ddrrr}{g_I} &
      D_J \arrow[bend left=20,swap, color=blue]{dr}{f_J}
      \arrow[bend left=40,color=red]{ddrr}{g_J} & & \\
      & & {\color{blue}U} \arrow[dashed]{dr}{h}& \\
      & & & {\color{red}V}
    \end{tikzcd}
\]
%% TODO: example drawing

\end{document}

%%% Local Variables:
%%% mode: latex
%%% TeX-master: t
%%% End:
