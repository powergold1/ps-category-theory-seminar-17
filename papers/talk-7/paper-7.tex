\def\pathToRoot{../../}\documentclass[a4paper,oneside]{scrbook}
\usepackage[dvipsnames]{xcolor}
\input{\pathToRoot headers/sharedHeader}

\usepackage{bm}
\usepackage{upgreek}
\usepackage{titlesec}

\geometry{margin=3cm}

\titleformat{\chapter}[display]
{\normalfont\LARGE\bfseries}{\chaptertitlename~\thechapter}{4pt}{\LARGE}
\titlespacing*{\chapter}{0pt}{2pt}{10pt}



% Environments theorem, lemma, definition and corollary.
% They all use the counter from theorem, except remark which has no counter
\newtheorem{theorem}{Theorem}[chapter]
\newtheorem{lemma}[theorem]{Lemma}
\newtheorem{corollary}[theorem]{Corollary}
\newtheorem{fact}[theorem]{Fact}
\newtheorem{example}[theorem]{Example}
\theoremstyle{definition}
\newtheorem{definition}[theorem]{Definition}
\newtheorem*{remark}{Remark}

\usepackage{charter}
\fontfamily{bch}\selectfont



%%% Local Variables:
%%% mode: latex
%%% TeX-master: "../papers/talk-7/paper-7"
%%% End:


\title{General limits}
\author{Leonhard Staut}

\begin{document}

%% TODO: consistent notation, DI D(I) D_I, also for maps D_m Dm

\maketitle
\chapter*{Introduction}
We have seen three special structures in categories.\\[1em]
\begin{minipage}{.35\linewidth}
  \begin{tikzcd}
    & V \arrow[bend right=40]{ddl} \arrow[bend left=40]{ddr} \arrow[dashed]{d} &\\
    & U \arrow{dl} \arrow{dr} &\\
    \bullet &&\bullet
  \end{tikzcd}
\end{minipage}%
\begin{minipage}{.35\linewidth}
  \begin{tikzcd}
    V
    \arrow[bend left=40]{drr}
    \arrow[bend right=40]{ddr}
    \arrow[dashed,swap]{dr}
    && \\
    &U
    \arrow{r}
    \arrow{d}
    &\bullet \arrow{d} \\
    & \bullet \arrow{r} & \bullet
  \end{tikzcd}
\end{minipage}%
\begin{minipage}{.3\linewidth}
  \begin{tikzcd}
    V \arrow[dashed]{dd} \arrow{dr}&& \\
    & \bullet \arrow[shift left]{r} \arrow[shift right]{r} & \bullet \\\
    U \arrow{ur} &&
  \end{tikzcd}
\end{minipage}\\[2em]
These three examples follow a pattern.
In each example there is an object $U$ with some arrows. This object $U$
has a universal property which states that any other object $V$ which has arrows
similar to $U$
factors uniquely through $U$, that is, there is a unique morphism $V \to U$,
such that the resulting triangles commute.\\
Products, pullbacks, and equalizers are instances of a more general concept called
limit which expresses the common pattern between these examples.\\
%% TODO: possibly give further applications of limits too
%% TODO: maybe some more intuition what limits are about

{
  %% Nice hack to get rid of page break just for the introduction chapter
  \renewcommand{\clearpage}{}
  \chapter{Definition}
}
In category theory we take limits of diagrams.
Therefore we first make the notion of a diagram more precise.

\begin{definition}
  Let $\bm{I}, \mathscr{C}$ be categories.
  A \textbf{diagram} is a functor $D \from \bm{I} \to \mathscr{C}$.
  The category $\bm{I}$ is called the \textbf{shape} of the diagram.
\end{definition}
\begin{example}
  \label{diagramexample}
  This is an example of a diagram of shape $\bm I$. \\[1cm]
  \begin{minipage}{0.3\linewidth}
    \hspace{2cm}
    \begin{tikzcd}
      & \bullet \arrow{d} \\
      \bullet\arrow{r} & \bullet
    \end{tikzcd}\\
  \end{minipage}%
  \begin{minipage}{0.2\linewidth}
    $\qquad\quad\overset{D}{\Longrightarrow}$
  \end{minipage}%
  \begin{minipage}{0.1\linewidth}
    \begin{tikzpicture}
      \draw(1,0) circle (1.5cm);
      \begin{tikzcd}
        & \bullet \arrow{d} \\
        \bullet\arrow{r} & \bullet
      \end{tikzcd}
    \end{tikzpicture}
  \end{minipage}\\[2em]
  \begin{minipage}{0.3\linewidth}
    \hspace{2cm}
    $\qquad \bm I$
  \end{minipage}%
  \begin{minipage}{0.2\linewidth}
    \
  \end{minipage}%
  \begin{minipage}{0.3\linewidth}
    \hspace{2cm}
    $\qquad \mathscr C$
  \end{minipage}\\[1em]
\end{example}

The functor $D$ yields
a collection of objects and morphisms in the category $\mathscr C$
that essentially looks like $\bm I$.
We also say $D$ \textbf{identifies} diagrams of shape $\bm I$.
The shape category has no further meaning. Its only purpose
is to define how the diagram is supposed to look like.
The reason we use an intermediate
category $\bm I$ to define the shape is that
there might be many parts in $\mathscr C$ which have the shape $\bm I$,
especially when $\bm I$ is so small.
We therefore get all different diagrams of that shape by considering
all different functors $D$.
Diagrams do not necessarily have to be small. Their shape can even be infinite.
However most interesting diagrams tend to be very small.\\
We are typically interested in diagrams of a specific category,
$\mathscr C$ in the example. Therefore we usually leave out $\bm I$ and $D$,
and we just draw the identified part of $\mathscr C$ instead.

\begin{definition}
  Let $D \from \bm{I} \to \mathscr{C}$ be a diagram.
  A \textbf{cone} $U$ over $D$ is an object $U \in \mathscr{C}$
  with maps $U \overset{f_I}{\rightarrow} D_I, \forall\ I \in \bm{I}$, such that
  $\forall\ I \overset{m}{\to} J \in \bm{I}$, the following triangle commutes:
  \[
    \begin{tikzcd}
      & U \arrow[dl, swap, "f_I"] \arrow{dr}{f_J}&\\
      D_I \arrow[rr, swap, "D(m)"] & & D_J\\
    \end{tikzcd}
  \]
  That is $f_J = D(m) \circ f_I$.\\
  For a cone we write $(U \overset{f_I}{\to} D(I))_{I\in \bm{I}}$.
  We call $U$ the \textbf{vertex}, and $f_I$ the \textbf{projection maps} of the cone.
\end{definition}
\begin{remark}
  When it is clear from the context that $U$ is the vertex of a cone, then we use just
  the vertex $U$
  to refer to the whole cone.
\end{remark}
%% TODO: Notation DI, D(I), D_I ??
\begin{example}
  A cone over the diagram of
  \hyperref[diagramexample]{Example \ref*{diagramexample}} looks like this.
  \[
    \begin{tikzcd}
      U \arrow[bend right=30]{ddr}{f_I} \arrow[bend left=30]{drr}{f_k} \arrow[bend left=20, swap]{ddrr}{f_J}&   &\\
      & & D_K \arrow{d}{D_n} \\
      & D_I \arrow{r}{D_m} & D_J
    \end{tikzcd}
  \]
  The commuting condition says that $D_m \circ f_I = f_J$ and $D_n \circ f_K = f_J$.
\end{example}

In other words a cone is an object that covers the diagram with projection maps,
such that all triangles commute.\\
Now we define what it means for a cone to be universal. Intuitively,
a cone is universal if it
is universally the closest to the diagram,
and that all other cones factor through it uniquely.

\begin{definition}
  A cone $(U \overset{f_I}{\to} D(I))_{I\in \bm{I}}$ is \textbf{universal}, if
  for all other cones $(V \overset{g_I}{\to} D(I))_{I\in \bm{I}}$ there is a unique
  map $h \from V \to U$, such that for all $I \in \bm{I}$
  the following triangles commute.
  \[
    \begin{tikzcd}
      V \arrow[dr, swap, "g_I"] \arrow{r}{h} & U \arrow{d}{f_I} \\
      & D_I
    \end{tikzcd}
  \]
  That is $g_i = f_i \circ h$.
\end{definition}

In other words, any cone over $D$ has maps into $D$, and a cone $U$ is universal if
any other cone $V$ has a unique morphism to $U$.
The commuting condition states that for any cone
the projection maps $g_I$ are equivalent
to the map that is obtained by composing the unique limit map $h$ with the
projection maps of the limit $f_I$.

\begin{example}
    \[
    \begin{tikzcd}
      {\color{red}V}
      \arrow[bend right=40,color=red]{dddrr}{g_I}
      \arrow[bend left=40, color=red]{ddrrr}{g_k}
      \arrow[bend left=30, color=red, swap]{dddrrr}{g_J}
      \arrow[dashed,swap]{dr}{h}
      & & & \\
      & {\color{blue}U}
      \arrow[bend right=30,color=blue]{ddr}{f_I}
      \arrow[bend left=30, color=blue]{drr}{f_k}
      \arrow[bend left=20, color=blue, swap]{ddrr}{f_J}&   &\\
      & & & D_K \arrow{d}{D_n} \\
      & & D_I \arrow{r}{D_m} & D_J
    \end{tikzcd}
  \]
  The universal cone {\color{blue}U}
  is drawn in blue and the other cone {\color{red}V} is drawn in red.
  Both cones obviously have to satisfy the commuting condition for cones.
  Additionally the commuting condition for universal cones states that
  $g_I = f_I \circ h$, $g_J = f_J \circ h$, and $g_K = f_k \circ h$.
  Therefore this example commutes in its entirety.
\end{example}

\begin{definition}
  A \textbf{limit} is a universal cone over a diagram.
\end{definition}

A limit of a diagram is therefore characterized by the kind of diagram
over which it is a universal cone, meaning we get different instances of limits
by choosing diagrams of different shapes.

\begin{example}
  Consider a diagram of the following shape
  \begin{tikzcd}
    \bullet &\bullet
  \end{tikzcd},
  which is just two objects without non-trivial arrows.
  A limit of this diagram looks like this.
  \[
    \begin{tikzcd}
      & {\color{red} V}
      \arrow[bend right=40,color=red, swap]{ddl}{g_I}
      \arrow[bend left=40,color=red]{ddr}{g_J}
      \arrow[dashed]{d}{h} &\\
      & {\color{blue} U}
      \arrow[color=blue, swap]{dl}{f_I}
      \arrow[color=blue]{dr}{f_J} &\\
      I && J
    \end{tikzcd}
  \]
  This is the ordinary binary product.
  The commuting condition for limits states that
  $f_I \circ h = g_I$, and $f_J \circ h = g_J$, which
  is exactly the commuting condition for products.
  We have therefore verified that the product is
  indeed an instance of the general limit definition.
  Using discrete categories with more than two objects
  we get products of arbitrary size.
  \[
    \begin{tikzcd}
      && {\color{red} V}
      \arrow[bend right=40,color=red]{ddll}
      \arrow[bend right=40,color=red]{ddl}
      \arrow[bend left=40,color=red]{ddrr}
      \arrow[bend left=40,color=red]{ddr}
      \arrow[dashed]{d} &&\\
      && {\color{blue} U}
      \arrow[color=blue]{dl}
      \arrow[color=blue]{dll}
      \arrow[color=blue]{dr}
      \arrow[color=blue]{drr} &&\\
      \bullet & \bullet & \ldots & \bullet &\bullet
    \end{tikzcd}
  \]  
\end{example}

\begin{example}
  \label{terminallimit}
  If we take the empty category as the shape for a diagram, then a cone
  over that diagram is just a single object $U$ that has no other required arrows.
  Then the universal property for limits states that every other cone, i.e. an object $V$,
  has a unique map $V \to U$. A limit of the empty diagram is therefore just a terminal object,
  and terminal objects are therefore another instance of the general limit definition.
\end{example}

\begin{example}
  Consider the following diagram.
  \begin{tikzcd}
    I \arrow{r}{m} & J & K \arrow[swap]{l}{n}
  \end{tikzcd}
  A cone over this diagram looks like this:
  \[
    \begin{tikzcd}
      U
      \arrow[]{r}{\ \ \ f_K}
      \arrow[swap]{rd}{\ f_J}
      \arrow[swap]{d}{f_I}
      &K \arrow{d}{n} \\
      I \arrow[swap]{r}{m} & J
    \end{tikzcd}
  \]
  The commuting condition for cones states however that
  $f_J = m\circ f_I$, and $f_J = n\circ f_K$. Since $f_J$ is uniquely determined
  by composition of other arrows, we do not need to draw it and leave it implicit.
  With this in mind a limit of this diagram looks like this:
  \[
    \begin{tikzcd}
      {\color{red}V}
      \arrow[bend left=50,color=red]{drr}{g_K}
      \arrow[bend right=50,color=red,swap]{ddr}{g_I}
      \arrow[dashed,swap]{dr}{h}
      && \\
      &{\color{blue}U}
      \arrow[color=blue]{r}{f_K}
      \arrow[color=blue, swap]{d}{f_I}
      &K \arrow{d}{n} \\
      & I \arrow[swap]{r}{m} & J
    \end{tikzcd}
  \]
  The commuting condition for limits results in exactly the known commuting condition for
  pullbacks. We have therefore verified that pullbacks are a valid instance of the general limit definition.
  It follows similarly that equalizers are limit instances over diagrams of the shape
  \begin{tikzcd}
    I \arrow[r,shift left, "m"] \arrow[r,shift right,swap,"n"] & J
  \end{tikzcd}.
\end{example}

\chapter{Building new limits}
It is easy to see that limits do not necessarily exists in all categories.
Sometimes we want to show that a category has certain limits,
but it might be difficult to show for this category that the limits
of interest exist.\\
In this chapter we show the following:\\
\hyperref[equiinstance]{(\ref*{equiinstance})} A category $\mathscr C$ has finite products and equalizers iff
it has pullbacks and a terminal
object, that is,
the limits on the right hand side can be constructed from the left hand side
and vice versa.\\
\hyperref[arbitrarylimits]{(\ref*{arbitrarylimits})}
If a category $\mathscr C$ has products and equalizers, then it has all limits, that is,
limits of an arbitrary diagram can be constructed using these two instances.

\section{Equivalence of instances}
\label{equiinstance}

In this section we establish that the instances of limits we have seen are related
in a way that allows us to show how to construct these instances using others.
This implies that if we know that a category has some of the right instances
we can construct other limits for free using only category theoretical arguments.

\begin{fact}
  \label{pullbackconstruction}
  Let $\mathscr C$ be a category with finite products and equalizers.
  Then we can construct pullbacks.
\end{fact}
\begin{proof}
We start with the diagram for pullbacks.
\[
  \begin{tikzcd}
    & K \arrow{d}{n} \\
    I \arrow[swap]{r}{m} & J
  \end{tikzcd}
\]
We need to construct a limit of this diagram.
First we the product of $I$ and $K$. This product exists by assumption.
\[
  \begin{tikzcd}
    I \times K \arrow{r}{p_K} \arrow{d}{p_I} & K \arrow{d}{n} \\
    I \arrow[swap]{r}{m} & J
  \end{tikzcd}
\]
This looks already similar to a pullback.
However, we need to ensure the commuting condition.
For this purpose we use an equalizer which also exists by assumption.
\[
  \begin{tikzcd}
    E \arrow{r}{e} &
    I \times K
    \arrow[r,shift right,swap, "m \circ p_I"] \arrow[r,shift left, "n \circ p_K"]
    & J \\
  \end{tikzcd}
\]
We now unfold this drawing.
\[
  \begin{tikzcd}
    E \arrow[swap]{d}{p_I \circ e} \arrow{r}{p_K \circ e} & K \arrow{d}{n} \\
    I \arrow[swap]{r}{m} & J
  \end{tikzcd}
\]
$E$ satisfies the commuting and universality conditions of limits
since it is already an equalizer.
Therefore it is a pullback.
\end{proof}
\begin{fact}
  \label{terminalproduct}
  Let $\mathscr C$ be a category with finite products and equalizers.
  Then we have a terminal object.
\end{fact}
\begin{proof}
  Recall a terminal object is a limit of the empty diagram as mentioned in
  \hyperref[terminallimit]{Example \ref*{terminallimit}}.
  Since we have finite products we have in particular the empty product
  which is nothing else than a limit of the empty diagram.
\end{proof}

\begin{fact}
  \label{pullbackproduct}
  Let $\mathscr C$ be a category with pullbacks and a terminal object $1$.
  Then we can construct finite products.
\end{fact}
\begin{proof}
Finite products can be constructed
with the terminal object and binary products. For any object
$A$, unary products are
constructed using the binary product $A \times 1$.
The empty product is just the terminal object $1$ as discussed in
\hyperref[terminalproduct]{Fact \ref*{terminalproduct}}.
All other finite products can be constructed by cascading binary products.
Therefore we only show how to construct binary products.
Consider following the diagram:
\[
  \begin{tikzcd}
    & B \arrow{d} \\
    A \arrow{r} & 1
  \end{tikzcd}
\]
The pullback $U$ exists by assumption.
\[
  \begin{tikzcd}
    U \arrow{r}{p_B} \arrow{d}{p_A} & B \arrow{d}{f} \\
    A \arrow{r}{g} & 1
  \end{tikzcd}
\]
We used the suggestive names $p_A$ and $p_B$ since it
is easy to see that $U$ is already the product of $A$ and $B$ by
simply leaving out $f$ and $g$. Their existence does not matter for the product.
In general, a pullback
is like a binary product with some additional structure.
\end{proof}

\begin{fact}
  Let $\mathscr C$ be a category with pullbacks and a terminal object $1$.
  Then we can construct equalizers.
\end{fact}
\begin{proof}
This construction uses the fact that $\mathscr C$ has products
as shown in \hyperref[pullbackproduct]{Fact \ref*{pullbackproduct}}.
Consider the following diagram.
\[
  \begin{tikzcd}
    & B \arrow{d}{\langle 1_B, 1_B \rangle} \\
    A \arrow{r}{\langle f,g \rangle} & B \times B
  \end{tikzcd}
\]
The product $B \times B$ exists by construction.
We then have the pullback $E$, such that the following diagram commutes.
\[
  \begin{tikzcd}
    E \arrow{r}{h} \arrow{d}{e} & B \arrow{d}{\langle 1_B, 1_B \rangle} \\
    A \arrow{r}{\langle f,g \rangle} & B \times B
  \end{tikzcd}
\]
Since $E$ is a pullback the square above commutes and
we get $f \circ e = h = g \circ e$, and $E$ is therefore
indeed an equalizer of $f$ and $g$.
\end{proof}

\section{Construction of arbitrary limits}
\label{arbitrarylimits}
We now study how arbitrary limits can be constructed using only a
few specific instances.
The approach is similar to the construction of pullbacks in
\hyperref[pullbackconstruction]{Fact \ref*{pullbackconstruction}}.

\begin{fact}
  Let $\mathscr C$ be a category that has all products and equalizers.
  Let $D : \bm I \to \mathscr C$ be any diagram.
  Then we can construct a limit of $D$.
\end{fact}
\begin{proof}
We use the following construction:
\[
  \begin{tikzcd}
    E \arrow{r}{e} &
    \displaystyle \prod_{I \in \bm{I}} D(I)
    \arrow[r,shift left, "s"] \arrow[r,shift right,swap,"t"] &
    \displaystyle \prod_{K \overset{n}{\to} J \in \bm{I}} D(J)
  \end{tikzcd}
\]
with the maps $s, t$ that have components $s_n, t_n$ for all $K \overset{n}{\to} J \in \bm{I}$:
\begin{align*}
  s_n &= Dn \circ p_K\ \hat{=} \prod_{I \in \bm{I}} D(I)
        \overset{p_K}{\rightarrow} D_K \overset{Dn}{\rightarrow} D_J \\
  t_n &= p_J\ \hat{=} \prod_{I \in \bm{I}} D(I)
        \overset{p_J}{\rightarrow} D_J
\end{align*}
All products exist by assumption.
To clarify, the product over all maps $K \overset{n}{\to} J \in \bm{I}$
has components $p_n = J$ for each map $n$.\\
We take $E$ as the equalizer of $s$ and $t$ which also exists by assumption.
We therefore have $s \circ e = t \circ e$.
The maps $s$ and $t$ have components. Therefore the equality has to hold for all
components and we get
\begin{align*}
  s \circ e &= t \circ e\\
  s_n \circ e &= t_n \circ e, \quad \forall\ K \overset{n}{\rightarrow} J \in \bm I \\
  Dn \circ p_K \circ e &= p_J \circ e
\end{align*}
The goal is to construct a limit.
We do this by constructing a cone and showing that it is universal.
For the construction of a cone
we need a vertex and projection maps into every object of the diagram,
such that the commuting condition of cones is satisfied.
We take $E$ as the vertex and
$\uprho_I \coloneqq p_I \circ e, \forall\ I \in \bm I$
as the projection maps of the cone.
Clearly, we have all necessary projection maps and we only need to verify
the commuting condition of cones.
We have all maps $K \overset{n}{\rightarrow} J \in \bm I$
\begin{align*}
  Dn \circ p_K \circ e &= p_J \circ e \\
  Dn \circ \uprho_K &= \uprho_J
\end{align*}
which corresponds to the commuting diagram
\[
  \begin{tikzcd}
    E \arrow{r}{\uprho_K} \arrow{dr}[swap]{\uprho_J} & D_K \arrow{d}{Dn}\\
    & D_J
  \end{tikzcd}
\]
This is exactly the commuting condition we need for a cone.\\
Since any other cone over $D$
is also a fork of $s$ and $t$
it follows that the equalizer $E$ is also a universal cone
and therefore a limit of $D$.
\end{proof}
\begin{remark}
  We can even construct limits of infinite diagrams
  because we assumed that $\mathscr C$ has all products.
  If $\mathscr C$ only has finite products, then we can
  use this construction for
  limits of finite diagrams.\\
  Using \hyperref[equiinstance]{Section \ref*{equiinstance}}
  we can alternatively require pullbacks and a terminal object instead of finite products.
\end{remark}

\chapter{Colimits}
To define colimits we dualize the concept of limits.
\begin{definition}
  Let $D : \bm I \to \mathscr C$ be a diagram.
  A \textbf{cocone} $U$ over $D$ is an object $U \in \mathscr{C}$
  with maps $D_I \overset{ f_I}{\rightarrow} U,\ \forall\ I \in \bm{I}$, such that
  $\forall\ I \overset{m}{\to} J \in \bm{I}$, the following triangle commutes:
  \[
    \begin{tikzcd}
      D_I \arrow[rr, "D(m)"]\arrow[dr, swap, "f_I"] & & D_J \arrow{dl}{f_J}\\
      & U  &
    \end{tikzcd}
  \]
  That is $f_I = f_J \circ D(m)$.
\end{definition}
A cone has maps from the vertex into the diagram,
while a cocone has maps from the diagram into the vertex.
\begin{definition}
  A cocone $(D(I)\overset{f_I}{\to} U)_{I\in \bm{I}}$ is \textbf{universal}, if
  for all other cones $(D(I) \overset{g_I}{\to} V)_{I\in \bm{I}}$ there is a unique
  map $h \from U \to V$, such that for all $I \in \bm{I}$
  the following triangles commute.
  \[
    \begin{tikzcd}
      D_I \arrow[dr, swap, "g_I"] \arrow{r}{f_I} & U \arrow{d}{h} \\
      & V
    \end{tikzcd}
  \]
  That is $g_i = f_i \circ h$.
\end{definition}

\begin{definition}
  A \textbf{colimit} is a universal co-cone over a diagram.
\end{definition}
%% TODO: complete definition

\[
    \begin{tikzcd}
      & D_K \arrow{d}{D_n}
      \arrow[bend left=30,color=blue]{ddr}{f_K}
      \arrow[bend left=50,color=red]{dddrr}{g_K} & & \\
      D_I \arrow{r}{D_m}
      \arrow[bend right=30,color=blue]{drr}{f_I}
      \arrow[bend right=40,color=red]{ddrrr}{g_I} &
      D_J \arrow[bend left=20,swap, color=blue]{dr}{f_J}
      \arrow[bend left=40,color=red]{ddrr}{g_J} & & \\
      & & {\color{blue}U} \arrow[dashed]{dr}{h}& \\
      & & & {\color{red}V}
    \end{tikzcd}
\]
%% TODO: example drawing

\end{document}

%%% Local Variables:
%%% mode: latex
%%% TeX-master: t
%%% End:
