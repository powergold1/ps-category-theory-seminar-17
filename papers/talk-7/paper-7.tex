\def\pathToRoot{../../}\documentclass[a4paper,oneside]{scrbook}
\usepackage[dvipsnames]{xcolor}
\input{\pathToRoot headers/sharedHeader}

\usepackage{bm}
\usepackage{upgreek}
\usepackage{titlesec}

\geometry{margin=3cm}

\titleformat{\chapter}[display]
{\normalfont\LARGE\bfseries}{\chaptertitlename~\thechapter}{4pt}{\LARGE}
\titlespacing*{\chapter}{0pt}{2pt}{10pt}



% Environments theorem, lemma, definition and corollary.
% They all use the counter from theorem, except remark which has no counter
\newtheorem{theorem}{Theorem}[chapter]
\newtheorem{lemma}[theorem]{Lemma}
\newtheorem{corollary}[theorem]{Corollary}
\newtheorem{fact}[theorem]{Fact}
\newtheorem{example}[theorem]{Example}
\theoremstyle{definition}
\newtheorem{definition}[theorem]{Definition}
\newtheorem*{remark}{Remark}

\usepackage{charter}
\fontfamily{bch}\selectfont



%%% Local Variables:
%%% mode: latex
%%% TeX-master: "../papers/talk-7/paper-7"
%%% End:


\title{General limits}
\author{Leonhard Staut}

\begin{document}

\maketitle
\chapter{Introduction}
We have seen examples of limits.\\
Insert examples here.\\
Now we generalize them...
\chapter{Definition}

\begin{definition}
  Let $\bm{I}, \mathscr{C}$ be categories.
  A \textbf{diagram} is a functor $D \from \bm{I} \to \mathscr{C}$.
  The category $\bm{I}$ is called the \textbf{shape} of the diagram.
\end{definition}

%% TODO: explain reasoning behind this definition and give example

\begin{definition}
  Let $D \from \bm{I} \to \mathscr{C}$ be a diagram.
  A \textbf{cone} $U$ over $D$ is an object $U \in \mathscr{C}$
  with maps $f_I, \forall\ I \in \bm{I}$, such that
  $\forall\ I \overset{m}{\to} J \in \bm{I}$, the following triangle commutes:
  \[
    \begin{tikzcd}
      & U \arrow[dl, swap, "f_I"] \arrow{dr}{f_J}&\\
      D_I \arrow[rr, swap, "D(m)"] & & D_J\\
    \end{tikzcd}
  \]
  That is $f_J = D(m) \circ f_I$.\\
  For a cone we write $(U \overset{f_I}{\to} D(I))_{I\in \bm{I}}$.
  We call $U$ the \textbf{vertex}, and $f_I$ the \textbf{projection maps} of the cone.
\end{definition}
\begin{remark}
  When $U$ is the vertex of a cone, then we sometimes use just $U$ to refer to the whole cone
  for better readability.
\end{remark}

%% TODO: write a sentence like "In other words, ..." and give example. example might be tricky to draw

\begin{definition}
  A cone $(U \overset{f_I}{\to} D(I))_{I\in \bm{I}}$ is universal, if
  for all other cones $(V \overset{g_I}{\to} D(I))_{I\in \bm{I}}$ there is a unique
  map $h \from V \to U$, such that for all $I \in \bm{I}$ the following triangles commute.
  \[
    \begin{tikzcd}
      V \arrow[dr, swap, "g_I"] \arrow{r}{h} & U \arrow{d}{f_I} \\
      & D_I
    \end{tikzcd}
  \]
  That is $g_i = f_i \circ h$.
\end{definition}

In other words, any cone over $D$ has maps into $D$, and a cone $U$ is universal if
for any other cone $V$, you can first go to $U$, that is the unique morphism.
From $U$ you can go to $D_I$, and the commuting condition says that this is equal
to the morphism that immediately takes you from $V$ to $D_I$.

%TODO: give example. might be tricky to draw.

\chapter{Building new limits}
It is possible to build new limits...

\chapter{Colimits}
Just dualize it... 

\end{document}

%%% Local Variables:
%%% mode: latex
%%% TeX-master: t
%%% End:
